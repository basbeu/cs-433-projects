% TODO/Idea 
%
% From project 2 guidelines :
% TODO Solid comparison baselines supporting your claims
% TODO Reproducibility
% TODO Scientific novelty and creativity
%
% from feedback project 1 : 
% values of the hyperparameters that you found after cross-validation
% plot that shows the training/test curves with respect to the regularization parameters
% Bselines/crossvalidation/comparisons/bias-variance
% scientific evidence (explanation of why what you do is working) 
%
% other ideas : 
% IDEA on pourrait comparer la performance du unet vs CNN classique qui nous ont donné.

\documentclass[10pt,conference,compsocconf]{IEEEtran}

\usepackage{hyperref}
\usepackage{graphicx}	% For figure environment
\usepackage{amsmath}
\usepackage{amssymb}

\begin{document}
\title{Road Segmentation}

\author{
  Beuchat Bastien, Mamie Robin, Mion Jeremy\\
  \textit{CS-433: Machine Learning (Fall 2019),}\\
   \textit{École Polytechnique Fédérale de Lausanne, Switzerland}
}

\maketitle

\begin{abstract}

\end{abstract}


\section{Introduction}

\section{Methodology}

\section{Data Preparation}

\subsection{Data Parsing}

\section{Improvements}

\subsection{Special Case: the Mass} \label{mass}

\subsubsection{Variable Expansion}\label{variable expansion}


\subsubsection{Splitting on DER\_mass\_MMC} \label{split mass}

\subsection{Feature Expansion}

\subsubsection{Non Linear}

\subsubsection{Polynomial}

\subsection{Combining PRI\_jet\_num }


\section{Experiments}

\section{Conclusion}

\newpage

\begin{thebibliography}{9}
%A citer proprement 
% https://arxiv.org/pdf/1505.04597.pdf


\end{thebibliography}

\end{document}